\documentclass[11pt,twocolumn]{article}
\usepackage{setspace}
\usepackage[margin = 1in]{geometry}
\usepackage{graphicx}
\usepackage[space]{grffile}
\usepackage[export]{adjustbox}
\graphicspath{ {/home/ahaggard/Pictures/} }
\begin{document}
\begin{flushleft}
Austin Haggard
\end{flushleft}
\doublespace
Below I included all of the graphs asked for.  As you can see, all of the histograms, regardless of distribution, have the shape of a normal distribution.  This is due to the central limit theorem.  This theorem states that given a large number of random samples from a distribution with a well defined mean and variance will yield approximately the normal distribution.  You can see this in all of the graphs below.  I first plot the given distribution.  I then took a random sample of 4000 points.  The mean of this sample was then taken with three different values of n.  As you can see from these graphs they all look approximately normal.  These are all examples of the central limit theorem. 
\newpage
\begin{figure}[h!]
\includegraphics[width=0.5\linewidth]{R Graphics: Device 2 (ACTIVE)_001}
\centering
\caption{\texttt{Standard Normal Curve}}

\includegraphics[width=0.5\linewidth]{R Graphics: Device 2 (ACTIVE)_002}
\centering
\caption{\texttt{n = 15}}

\includegraphics[width=0.5\linewidth]{R Graphics: Device 2 (ACTIVE)_003}
\centering
\caption{\texttt{n = 35}}

\includegraphics[width=0.5\linewidth]{R Graphics: Device 2 (ACTIVE)_004}
\centering
\caption{\texttt{n = 100}}
\end{figure}

\begin{figure}[h!]
\includegraphics[width=0.5\linewidth]{R Graphics: Device 2 (ACTIVE)_005}
\centering
\caption{\texttt{Normal mean = 5 sd = 5}}

\includegraphics[width=0.5\linewidth]{R Graphics: Device 2 (ACTIVE)_006}
\centering
\caption{\texttt{n = 15}}

\includegraphics[width=0.5\linewidth]{R Graphics: Device 2 (ACTIVE)_007}
\centering
\caption{\texttt{n = 35}}

\includegraphics[width=0.5\linewidth]{R Graphics: Device 2 (ACTIVE)_008}
\centering
\caption{\texttt{n = 100}}
\end{figure}

\begin{figure}[h!]

\includegraphics[width=0.5\linewidth]{R Graphics: Device 2 (ACTIVE)_009}
\centering
\caption{\texttt{Uniform a = 0, b = 1}}

\includegraphics[width=0.5\linewidth]{R Graphics: Device 2 (ACTIVE)_010}
\centering
\caption{\texttt{n = 15}}

\includegraphics[width=0.5\linewidth]{R Graphics: Device 2 (ACTIVE)_011}
\centering
\caption{\texttt{n = 35}}

\includegraphics[width=0.5\linewidth]{R Graphics: Device 2 (ACTIVE)_012}
\centering
\caption{\texttt{n = 100}}
\end{figure}

\begin{figure}[h!]
\includegraphics[width=0.5\linewidth]{R Graphics: Device 2 (ACTIVE)_013}
\centering
\caption{\texttt{Pois lambda = 5}}

\includegraphics[width=0.5\linewidth]{R Graphics: Device 2 (ACTIVE)_014}
\centering
\caption{\texttt{n = 15}}

\includegraphics[width=0.5\linewidth]{R Graphics: Device 2 (ACTIVE)_015}
\centering
\caption{\texttt{n = 35}}

\includegraphics[width=0.5\linewidth]{R Graphics: Device 2 (ACTIVE)_016}
\centering
\caption{\texttt{n = 100}}
\end{figure}

\begin{figure}[h!]
\includegraphics[width=0.5\linewidth]{R Graphics: Device 2 (ACTIVE)_017}
\centering
\caption{\texttt{Exponential mu = 2}}

\includegraphics[width=0.5\linewidth]{R Graphics: Device 2 (ACTIVE)_018}
\centering
\caption{\texttt{n = 15}}

\includegraphics[width=0.5\linewidth]{R Graphics: Device 2 (ACTIVE)_019}
\centering
\caption{\texttt{n = 35}}

\includegraphics[width=0.5\linewidth]{R Graphics: Device 2 (ACTIVE)_020}
\centering
\caption{\texttt{n = 100}}
\end{figure}
\end{document}
