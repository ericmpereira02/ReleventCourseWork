%Lab 5 Report
%LabName

\documentclass{article}
\usepackage{setspace}
\usepackage[margin = 1in]{geometry}
%---------------------------------------------------------------------------
%	TITLE SECTION
%---------------------------------------------------------------------------
\title{Experiment 5 \\ Resistors in Series and Parallel}
\author{by \\ Austin Haggard}

\date{
	Experiment Performed: Feb 25 \\
	Report Submitted: March 4 \\[11pt]
	Lab Partner: \\ Mohit Chandiramani \\[11pt] 
	GSA: \\ C. Monroe
}
%---------------------------------------------------------------------------
\begin{document}
\maketitle
\thispagestyle{empty}
\newpage

\section*{Introduction:}
\setcounter{page}{1}
%---------------------------------------------------------------------------
%	INTRODUCTION... What we are measuring and what we are comparing it too 	/ dont use first person / audience is another student
%---------------------------------------------------------------------------
The purpose of this experiment was to measure the resistance created by different resistors.  The resistors were hooked up in different ways such as being in series and parallel.  We also calculated the voltage and current in some of these circuits.  These measurements were compared to the theoretical value and observed.  
%---------------------------------------------------------------------------
\newpage

%---------------------------------------------------------------------------
%	DATA
%---------------------------------------------------------------------------
%	On a seperate page normally
%---------------------------------------------------------------------------

\section*{Data Analysis:}
\setcounter{page}{3}
%---------------------------------------------------------------------------
%	DATA ANALYSIS.. All equations used
%---------------------------------------------------------------------------
	In this experiment we had to calculate a d and a sigma d to determine if the resistance statistically agrees with what is supposed to.  We will use the equation and sample calculation below:
	\[\sigma_d = \sqrt{\sigma_e^2 + \sigma_t^2}\]
	\[\sigma_d = \sqrt{11.312^2 + 37.5^2} = 39.169\]
	\[d = |e - t|\]
	\[d = 302 - 300 = 2\]
	We also used ohms law to determine the internal resistance of a voltmeter.  Sample calculation below:
	
	\[R = \frac{V}{I}\]
	\[R = \frac{2E-6}{21} = 9.5E-8\ ohms\]
	
	
%---------------------------------------------------------------------------
\newpage

\section*{Table of Results:}
\begin{center}
\begin{tabular}{|l|l|}
\hline
%--------------------------------------------------------------------------
%	TABLE OF RESULTS
%--------------------------------------------------------------------------
	Experimental: 302 & Theoretical: 300  \\ \hline
	Discrepancy: 2   & Percent Difference: .66\% \\
%---------------------------------------------------------------------------
\hline
\end{tabular}
\end{center}

\section*{Discussion:}
\doublespace
%---------------------------------------------------------------------------
%	DISCUSSION
%---------------------------------------------------------------------------
%---------------------------------------------------------------------------
%	short paragraph on physics of the experiment
%---------------------------------------------------------------------------
In this experiment we used an ammeter to measure the resistance of different resistors.  We also calculated resistance after finding the voltage and current in a circuit.  After calculating d and sigma d for the resistors it was found that this part of the lab was successful with a difference of 2 and a sigma d of 39.  When measuring current flow through points a, b, and c in figure 1 the data agrees with the expected results.  When in series all the values are the same.  When in parallel the current decreases.  When looking at the voltage reading on the power supply and the multimeter there is a very small difference.  The power supply would be more reliable because the voltmeter has the resistance of the wires to read the voltage through while the power supply is just displaying the exact amount of power that it is outputting.  When a resistor is in parallel the relationship between voltage drop is that they will all add up to the total voltage in the end.  When resistors are in series the voltage drop will be equal throughout all of the resistors.  When using an ammeter it should have a low internal resistance.  This is because it can then be set up in a series and not effect the amount of current flowing through it.  A voltmeter should have a high internal resistance because it needs to be set up in parallel.  

This lab contained both intrinsic error and random errors in measurement.  Due to small variables such as the resistance of the devices used, this caused differences that slightly skewed the results of the experiments.  This is a random error because it is unable to be fixed and it can either effect the results of the experiment in either a positive or negative direction.  The cause of the intrinsic error in this experiment was due to the devices used in the experiment.  None of these errors were large enough to cause a problem with the experiment because all of ours were successful.  

%---------------------------------------------------------------------------
%	SOURCES OF EXPERIMENTAL ERROR TO OCCUR ie.. each category, 			 		random\intrinsic and how they effected the results, and which error is 	the largest
%---------------------------------------------------------------------------
  
  
%---------------------------------------------------------------------------
%	State wether it was a success / compare theoreticle and experiemntal.  	Explain discrepencies larger than normal
%---------------------------------------------------------------------------


%---------------------------------------------------------------------------
\singlespace
\newpage

\section*{Conclusion:}
%---------------------------------------------------------------------------
%	CONCLUSION... two sentences summerize experiment.  State whether it 			was a success and compare experimental / theoretical results.
%---------------------------------------------------------------------------
In this experiment we measured resistances in multiple different circuits.  We also looked at the effects of having circuits in both series ad parallel.  In the first part of this experiment we had a theoretical value of 300 and an experimental value of 302.  This is a very close result and was a success with a percent difference of .66\%.
%--------------------------------------------------------------------------
\end{document}