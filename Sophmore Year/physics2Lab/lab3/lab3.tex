%Lab # Report
%LabName

\documentclass{article}
\usepackage{setspace}
\usepackage[margin = 1in]{geometry}
%---------------------------------------------------------------------------
%	TITLE SECTION
%---------------------------------------------------------------------------
\title{Experiment 3 \\ Capacitance and Dielectrics}
\author{by \\ Austin Haggard}

\date{
	Experiment Performed: Feb 4 2015 \\
	Report Submitted: Feb 11 2015\\[11pt]
	Lab Partner: \\ Jonathan Cirillo \\[11pt] 
	GSA: \\ C. Monroe
}
%---------------------------------------------------------------------------
\begin{document}
\maketitle
\thispagestyle{empty}
\newpage

\section*{Introduction:}
\setcounter{page}{1}
%---------------------------------------------------------------------------
%	INTRODUCTION... What we are measuring and what we are comparing it too 	/ dont use first person / audience is another student
%---------------------------------------------------------------------------
During this experiment we first measured the capacitance of the electrometer and its cable.  This was then compared to a theoretical value.  In the second part of the experiment we measured capacitors set up both in a series and in parallel.  In the third part of this experiment we used two parallel plates.  With them we moved them closer together and kept measuring it's voltage.  We used this information to calculate the dielectric constant.  We also compared the capacitance of the plates and compared it to the theoretical value and calculated the percent difference.   
%---------------------------------------------------------------------------
\newpage

%---------------------------------------------------------------------------
%	DATA
%---------------------------------------------------------------------------
%	On a seperate page normally
%---------------------------------------------------------------------------

\section*{Data Analysis:}
\setcounter{page}{3}
%---------------------------------------------------------------------------
%	DATA ANALYSIS.. All equations used
%---------------------------------------------------------------------------
In the first part of the lab we calculated the capacitence of the voltmeter by reading it directly from the device.  We then calculated the capacitance using the following sample equation below.
\[C_e = \frac{V_0*C_0}{V_e}-C_0 \] \[396.11 = \frac{8 * 155}{2.25} - 155\]
Then we need to subtract 25 from our answer which yields the capacitence of the cable.
\[371.11F = 396.11 - 25\]
In part two we needed to calculate the charge of capacitors in series.  To do that we used the following equation.
\[\sum_{i = 1}^{C_n}\frac{1}{C_i}\] \[.055F = \frac{1}{100} + \frac{1}{47} + \frac{1}{11.2}\]
In the third part of this experiment we first measured the diameter of the plates used and calculated the Area.  We then calculated the capacitance of the plates using the following equation.
\[C = \frac{\epsilon_{air} * A}{d}\]
\[4.93F = \frac{.370 * 1.00059}{.075}\]
After this our last calculation was the percent difference.  This is calculated using the following equation.
\[\frac{C_{th} - C_{ex}}{C_{th}} * 100\]
\[97\% = \frac{4.93 - .135}{4.93} * 100\]
%---------------------------------------------------------------------------
\newpage

\section*{Table of Results:}
\begin{center}
\begin{tabular}{|l|l|}
\hline
%--------------------------------------------------------------------------
%	TABLE OF RESULTS
%--------------------------------------------------------------------------
	Experimental: 371.11F | 9.99F | 0.135F & Theoretical: N/A | 10F | 4.93F N/A  \\ \hline
	Discrepancy: N/A | .01 | 4.795   & Percent Difference: N/A | N/A | 97.3 \% \\
%---------------------------------------------------------------------------
\hline
\end{tabular}
\end{center}

\section*{Discussion:}
\doublespace
%---------------------------------------------------------------------------
%	DISCUSSION
%---------------------------------------------------------------------------
%---------------------------------------------------------------------------
%	short paragraph on physics of the experiment
%---------------------------------------------------------------------------
All three of our experiments were failures.  During the first experiment we measured the capacitance of a meter and the wire connected to it by taking readings off a capacitor.  The second experiment we hooked different level capacitors in both series and parallel then measured the capacitance and compared it to an expected value.  The third experiment had two parallel plates at set distances that we eventually found the capacitance for.  
%---------------------------------------------------------------------------
%	SOURCES OF EXPERIMENTAL ERROR TO OCCUR ie.. each category, 			 		random\intrinsic and how they effected the results, and which error is 	the largest
%---------------------------------------------------------------------------
  During this experiment there was a very large amount of error.  The majority of the error came from the measuring tools due to the fact that they are not made to measure with the precision we wanted.  For two out of three of the capacitors in part two we couldn't get a reading on due to overflow.  Our multimeter would also spike to random voltages when touching nothing and you could hear something loose inside of it.  These errors would be considered random errors in measurement.  We don't know if the values will be higher or lower due to the problems we have with the measuring tools provided.  \par 
  this experiment would be considered unsuccessful.  Given that it was unsuccessful, we still gathered the results that we expected with the equipment provided.
  
%---------------------------------------------------------------------------
%	State wether it was a success / compare theoreticle and experiemntal.  	Explain discrepencies larger than normal
%---------------------------------------------------------------------------


%---------------------------------------------------------------------------
\singlespace
\newpage

\section*{Conclusion:}
%---------------------------------------------------------------------------
%	CONCLUSION... two sentences summerize experiment.  State whether it 			was a success and compare experimental / theoretical results.
%---------------------------------------------------------------------------
In conclusion, we measured capacitance  in a few different ways.  Each way gave us a little bit of a different look and provided different ways of understanding the theory.  Our experiment was a failure with a 97\% error.  The way this experiment was provided, that is about what we expected.  These errors were due to faults in the equipment given.
%--------------------------------------------------------------------------
\end{document}