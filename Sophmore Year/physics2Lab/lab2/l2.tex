%Lab # Report
%LabName

\documentclass{article}
\usepackage{setspace}
\usepackage[margin = 1in]{geometry}
%---------------------------------------------------------------------------
%	TITLE SECTION
%---------------------------------------------------------------------------
\title{Experiment 2 \\ Electric Potential and Field Mapping}
\author{by \\ Austin Haggard}

\date{
	Experiment Performed: Jan 28 2015 \\
	Report Submitted: Feb 4 2015 \\[11pt]
	Lab Partner: \\ Adam \\[11pt] 
	GSA: \\ C. Monroe
}
%---------------------------------------------------------------------------
\begin{document}
\maketitle
\thispagestyle{empty}
\newpage

\section*{Introduction:}
\setcounter{page}{1}
%---------------------------------------------------------------------------
%	INTRODUCTION... What we are measuring and what we are comparing it too 	/ dont use first person / audience is another student
%---------------------------------------------------------------------------
In this experiment we are measuring electric fields.  We do this by charging plates that are sitting in a thin layer of water using the HY3003D power supply. We then measure at two centimeter intervals around the entire field and create a 3D graph to display our findings.  We do this between two parallel charged plates and also around a charged line.
%---------------------------------------------------------------------------
\newpage

%---------------------------------------------------------------------------
%	DATA
%---------------------------------------------------------------------------
%	On a seperate page normally
%---------------------------------------------------------------------------

\section*{Data Analysis:}
\setcounter{page}{3}
%---------------------------------------------------------------------------
%	DATA ANALYSIS.. All equations used
%---------------------------------------------------------------------------
There are no calculations to analyse in this experiment  Appendix B section C states when experiments are qualitative that this section is omitted and the point value of the discussion section is weighted accordingly
%---------------------------------------------------------------------------
\newpage

\section*{Table of Results:}
\begin{center}
\begin{tabular}{|l|l|}
\hline
%--------------------------------------------------------------------------
%	TABLE OF RESULTS
%--------------------------------------------------------------------------
	Experimental: N/A & Theoretical: N/A  \\ \hline
	Discrepancy: N/A   & Percent Difference: N/A \\
%---------------------------------------------------------------------------
\hline
\end{tabular}
\end{center}

\section*{Discussion:}
\doublespace
%---------------------------------------------------------------------------
%	DISCUSSION
%---------------------------------------------------------------------------
%---------------------------------------------------------------------------
%	short paragraph on physics of the experiment
%---------------------------------------------------------------------------
During this experiment our goal is to analyse electric fields.  To do this we charged two parallel plates, and a charged line and set them in a small trap of water.  We then took measurements of the voltage at two centimeter intervals along the entire area.  This gave us a great range of charges.  This lets us see the way the electric field works by seeing how the field is charged at different points along the effected area.  When looking at equipotential field lines you'll see that they are always parallel to electric field lines.  If we were to look at a vector field diagram we would see all the directions for all the points in space compared to a field line diagram is shown emitting lines fro a specific point.Our results agreed with the theory because you can see the points on our 3D graph that lines up with the expected results.

%---------------------------------------------------------------------------
%	SOURCES OF EXPERIMENTAL ERROR TO OCCUR ie.. each category, 			 		random\intrinsic and how they effected the results, and which error is 	the largest
%---------------------------------------------------------------------------
  During our experiment our biggest error was while measuring the parallel plates.  Instead of recording the measurements made by pressing the "keep" button in longer pro we measured it just by looking at the numbers.  This rounded our numbers greatly and also gave a much less accurate value.  This error would be classified as random error in measurement because we can't read the calculation nearly as accurate as the computer and it can be either higher or lower than the actual value.  This error can be seen in our first graph because it looks much flater than it should be.  There should be a curve area around the top, and it is much smaller than it should be.  If we were able to rotate this graph it would most likely still be there, but it should still be a lot more prominent than it currently is.  Our second 3D graph looks great with a sharp peak as expected.
  
%---------------------------------------------------------------------------
%	State wether it was a success / compare theoreticle and experiemntal.  	Explain discrepencies larger than normal
%---------------------------------------------------------------------------
This experiment would be considered a success.  We don't have any hard numbers to compare in regard to theoretical and experimental values.  We do see that the information collected is in line with what is theoretically expected of electric fields.

%---------------------------------------------------------------------------
\singlespace
\newpage

\section*{Conclusion:}
%---------------------------------------------------------------------------
%	CONCLUSION... two sentences summerize experiment.  State whether it 			was a success and compare experimental / theoretical results.
%---------------------------------------------------------------------------
In this experiment we measured two different types of electric fields.  One with two parallel plates, and another with a charged line.  We then took measurements at set points and made it into a 3D graph.  Our experiment was a success even though we don't have any theoretical numbers to compare it too.  We produced a graph that is on par with what was expected from an electric field.
%--------------------------------------------------------------------------
\end{document}