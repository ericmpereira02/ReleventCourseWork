%Lab 12 Report
%Sound and Waves

\documentclass{article}
\usepackage{setspace}
\usepackage[margin = 1in]{geometry}
%---------------------------------------------------------------------------
%	TITLE SECTION
%---------------------------------------------------------------------------
\title{Experiment 12 \\ Sound and Waves}
\author{by \\ Austin Haggard}

\date{
	Experiment Performed: April 22th \\
	Report Submitted: April 29th \\[11pt]
	Lab Partner: \\ Mohit Chandirmani \\[11pt] 
	GSA: \\ C. Monroe
}
%---------------------------------------------------------------------------
\begin{document}
\maketitle
\thispagestyle{empty}
\newpage

\section*{Introduction:}
\setcounter{page}{1}
%---------------------------------------------------------------------------
%	INTRODUCTION... What we are measuring and what we are comparing it too 	/ dont use first person / audience is another student
%---------------------------------------------------------------------------
In this experiment we measured the fundamental frequency and compared it t a theoretical value.  We did this by using a Sonomenter.  We plucked a string on this device and measured its frequency using logger pro. We also calculated frequency using a string and a wave driver.
%---------------------------------------------------------------------------
\newpage

%---------------------------------------------------------------------------
%	DATA
%---------------------------------------------------------------------------
%	On a seperate page normally
%---------------------------------------------------------------------------

\section*{Data Analysis:}
\setcounter{page}{3}
%---------------------------------------------------------------------------
%	DATA ANALYSIS.. All equations used
%---------------------------------------------------------------------------
In this experiment we first need to calculate the theoretical value of frequency.  We can do this by using the following equation.
\[f = \frac{1}{\lambda}\sqrt{\frac{W}{\mu}}\]
\[55.1 = \frac{1}{2}\sqrt{\frac{98}{.00805}}\]
\\
We also had to calculate linear mass density.  We can do this by using the equation below:
\[\frac{W}{\lambda^2* f_e^2}\]
\[.007 = \frac{4.9}{.308^2*86^2}\]
\\
Sigma D and difference were not required for this lab.
%---------------------------------------------------------------------------
\newpage

\section*{Table of Results:}
\begin{center}
\begin{tabular}{|l|l|}
\hline
%--------------------------------------------------------------------------
%	TABLE OF RESULTS
%--------------------------------------------------------------------------
	Experimental: 61.04 & Theoretical: 60.43  \\ \hline
	Discrepancy: .61  & Percent Difference: N/A \\
%---------------------------------------------------------------------------
\hline
\end{tabular}
\end{center}

\section*{Discussion:}
\doublespace
%---------------------------------------------------------------------------
%	DISCUSSION
%---------------------------------------------------------------------------
%---------------------------------------------------------------------------
%	short paragraph on physics of the experiment
%---------------------------------------------------------------------------
In the first part of this experiment we used logger pro and a microphone to measure the frequency of a sound wave.  this wave was produced by plucking the string of a Sonometer.  In the second part of the experiment we produced a wave with a string and a vibration producing apparatus.  We adjusted the frequency until a clear wave was formed.  We did this 3 different times with different amount of anti nodes.  

%---------------------------------------------------------------------------
%	SOURCES OF EXPERIMENTAL ERROR TO OCCUR ie.. each category, 			 		random\intrinsic and how they effected the results, and which error is 	the largest
%---------------------------------------------------------------------------
 If we look at both parts of this experiment we can see that both types of waves are very similar.  The key difference between these two types of waves are that the sound waves are mechanical and are compression waves vs. a standard wave.  In the first part of this experiment there is a large chance of error.  To calculate the fundamental frequency we looked at the information produced by logger pro by hovering over a jump in the data. It is very easy to hover to far to the left of right of it and receive numbers that aren't accurate.  This would be considered random errors in measurement because it can be skewed in either a positive or negative direction.  In the second part of the experiment there is also a random error in measurement.  This is from different measurements made such as the frequency from the signal generator and wavelength.  This experiment would be considered a success.  a d vs sigma d comparison wasn't required, but this experiment would be considered a success because as stated in the manual the values differ by only a few percent.
  
%---------------------------------------------------------------------------
%	State wether it was a success / compare theoreticle and experiemntal.  	Explain discrepencies larger than normal
%---------------------------------------------------------------------------


%---------------------------------------------------------------------------
\singlespace
\newpage

\section*{Conclusion:}
%---------------------------------------------------------------------------
%	CONCLUSION... two sentences summerize experiment.  State whether it 			was a success and compare experimental / theoretical results.
%---------------------------------------------------------------------------
In conclusion, during this experiment we calculated the frequency of different waves.  We did this with sound waves and a wave made from a string and vibrating apparatus.  This experiment would be considered a success because the experimental values were very close to the theoretical values. 
%--------------------------------------------------------------------------
\end{document}