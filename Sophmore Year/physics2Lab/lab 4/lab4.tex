%Lab 4 Report
%LabName

\documentclass{article}
\usepackage{setspace}
\usepackage[margin = 1in]{geometry}
%---------------------------------------------------------------------------
%	TITLE SECTION
%---------------------------------------------------------------------------
\title{Experiment 4 \\ Ohms Law}
\author{by \\ Austin Haggard}

\date{
	Experiment Performed: 11FEB2015 \\
	Report Submitted: 25FEB2015\\[11pt]
	Lab Partner: \\ Nick Klofta \\[11pt] 
	GSA: \\ C. Monroe
}
%---------------------------------------------------------------------------
\begin{document}
\maketitle
\thispagestyle{empty}
\newpage

\section*{Introduction:}
\setcounter{page}{1}
%---------------------------------------------------------------------------
%	INTRODUCTION... What we are measuring and what we are comparing it too 	/ dont use first person / audience is another student
%---------------------------------------------------------------------------
In this experiment we are first measuring the resistance of our skin using a multimeter and will later use this value to calculate the amount of voltage that would be needed to cause harm using the D.C. power supply.  This part was done to show just how safe the equipment is.  We then set up light bulbs in both series and parallel and observed how this and the amount of power applied affected their brightness.  In the last part of this experiment we measured the potential at different lengths from a power source and observed how it was effected.
%---------------------------------------------------------------------------
\newpage

%---------------------------------------------------------------------------
%	DATA
%---------------------------------------------------------------------------
%	On a seperate page normally
%---------------------------------------------------------------------------

\section*{Data Analysis:}
\setcounter{page}{3}
%---------------------------------------------------------------------------
%	DATA ANALYSIS.. All equations used
%---------------------------------------------------------------------------
To calculate the force needed to apply 300 mA to the skin you would use the following equation.

\[V = I*r\]
this means that the voltage to be applied can be found below:

\[300 * 110 = 33000 V\]
Part 3 and 4 of the lab data was measured instead of calculated.  You can see these values on the data sheet. \\ \\
The graph labelled Distance vs Potential on the data sheet is showing how the Potential is decreasing as the distance is increasing along the meter stick.
%---------------------------------------------------------------------------
\newpage

\section*{Table of Results:}
\begin{center}
\begin{tabular}{|l|l|}
\hline
%--------------------------------------------------------------------------
%	TABLE OF RESULTS
%--------------------------------------------------------------------------
	Experimental: N/A & Theoretical: N/A  \\ \hline
	Discrepancy: N/A   & Percent Difference: N/A \\
%---------------------------------------------------------------------------
\hline
\end{tabular}
\end{center}

\section*{Discussion:}
\doublespace
%---------------------------------------------------------------------------
%	DISCUSSION
%---------------------------------------------------------------------------
%---------------------------------------------------------------------------
%	short paragraph on physics of the experiment
%---------------------------------------------------------------------------
During this experiment we observed the flow of power in circuits by using light bulbs.  We set up multiple types of circuits and adjusted the power to different voltages and observed how the lighting was effected.  We also measured how potential dissipates as it travels a distance. In part one of the lab some factors that could effect your skins resistance are things such as how calloused your hands are.  If you were to put 30 V into your hand using the skin resistance we calculated it would force .272 milli amps into your skin.  If you look at figure 6 a charge carrying 1 coulomb would carry the same charge throughout.  With two bulbs in a series you would have a full charge at the starting point and it would dissipate after the first bulb it passes through.  Potential values in regards to light bulbs have a direct correlation to how bright they are.  The larger the potential value, the brighter the bulb.  If you were to place three bulb in a series you could calculate their current by adding together all of the resistances under 1 and dividing the 1 by the answer.  If two bulbs were placed in a series the second bulb in line would be much dimmer than the first bulb.  In figure 7 at point 6 it would carry the same amount of charge as if it travelled through 1 light bulb.  In figure 8 in the lab manual, bulb B and C will be dim and A will be brighter.  In part 4 of this experiment the slope of the line shows the pace at which the potential decreases because it is set at a constant interval.  You can see in my drawing that potential energy decreases as the distance increases.

This experiment had a systematic errors in measurement.  This is due to using the multimeter to measure the potential which is not always accurate. This experiment would also be considered a success.  All of the measured values and observations were in line with the expected physics.  There was no theoretical values to compare too so we can't have solid numbers to measure the success or failure.   

%---------------------------------------------------------------------------
%	SOURCES OF EXPERIMENTAL ERROR TO OCCUR ie.. each category, 			 		random\intrinsic and how they effected the results, and which error is 	the largest
%---------------------------------------------------------------------------
  
  
%---------------------------------------------------------------------------
%	State wether it was a success / compare theoreticle and experiemntal.  	Explain discrepencies larger than normal
%---------------------------------------------------------------------------


%---------------------------------------------------------------------------
\singlespace
\newpage

\section*{Conclusion:}
%---------------------------------------------------------------------------
%	CONCLUSION... two sentences summerize experiment.  State whether it 			was a success and compare experimental / theoretical results.
%---------------------------------------------------------------------------
In this experiment we built circuits with light bulbs and applied varying voltages to them.  After this we measured how much potential dissipates over a distance.  This experiment would be considered a success.  Our measured values and observations were in line with the expected physics of the experiment.  There was no theoretical value to compare too so we can not have a percent difference.
%--------------------------------------------------------------------------
\end{document}