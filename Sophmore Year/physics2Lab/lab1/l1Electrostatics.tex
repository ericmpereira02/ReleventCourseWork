%Lab 1 Report
%Electrostatics
\documentclass{article}
\usepackage{setspace}
%---------------------------------------------------------------------------
%	TITLE SECTION
%---------------------------------------------------------------------------
\title{Experiment 1 \\ Electrostatics}
\author{by \\ Austin Haggard}

\date{
	Experiment Performed: 1/14/15 \\
	Report Submitted: 1/28/15 \\[11pt]
	Lab Partner: \\ Jacob Kenny \\[11pt] 
	GSA: \\ C. Monroe
}
%---------------------------------------------------------------------------
\begin{document}
\maketitle
\thispagestyle{empty}
\newpage

\section*{Introduction:}
\setcounter{page}{1}
%---------------------------------------------------------------------------
%	INTRODUCTION
%---------------------------------------------------------------------------
The purpose of this experiment is to charge and discharge two wands and make observations about their electrostatic charge.  These charges are then recorded and compared.  This is done by rubbing the two wands together, along with charging them through a conduction sphere. 
%---------------------------------------------------------------------------
\newpage

%---------------------------------------------------------------------------
%	DATA
%---------------------------------------------------------------------------
%	On a seperate page normally
%---------------------------------------------------------------------------

\section*{Data Analysis:}
\setcounter{page}{3}
%---------------------------------------------------------------------------
%	DATA ANALYSIS
%---------------------------------------------------------------------------
There are no calculations to analyse in this experiment  Appendix B section C states when experiments are qualitative that this section is omitted and the point value of the discussion section is weighted accordingly.
%---------------------------------------------------------------------------
\newpage

\section*{Table of Results:}
\begin{center}
\begin{tabular}{|l|l|}
\hline
%--------------------------------------------------------------------------
%	TABLE OF RESULTS
%--------------------------------------------------------------------------
	Experimental: N/A & Theoretical: N/A  \\ \hline
	Discrepancy: N/A   & Percent Difference: N/A \\
%---------------------------------------------------------------------------
\hline
\end{tabular}
\end{center}

\section*{Discussion:}
\doublespace
%---------------------------------------------------------------------------
%	DISCUSSION
%---------------------------------------------------------------------------
During the first part of this experiment we rubbed two charging wands together and then put them separately into the Faraday pail.  This allows us to measure the voltage through the attached electro meter.  We then record different variations of this by putting in both at the same time touching the cage, along with both at the same time touching each other.  In situations where we touch the wand to the inside of the cage, the meter continues reading the same value.  This is because the charged electrons stick to the cage leaving it charged when touched.  When you reverse the wand used, the voltages are about the same but they are positive instead of negative.  When both wands are put into the cage, theoretically the value should end up being zero because their charges should cancel each other out.  There was error in our experiment which caused one of the wands to overpower the other, which I will talk about further in the lab discussion.    \par

During part two of our experiment we set up two conductive spheres and collected charged particles from the sphere.  We then measured their charge in the Faraday pail.  We do this directly from the charged sphere, and also from an uncharged sphere placed near the charged sphere.  We also measure the charge from different locations on the sphere to see the difference.  The inside location of the sphere has a much stronger charge than the outside. \par 

In part three of this experiment you took samples from two sides of an oblong sphere along with samples from a hollowed sphere.  When taking samples from the center of the the hollow sphere there was no charge because the charges repel and are distributed on the outside of the sphere as far away as possible.  This shows that static charge is only present on the outside of a surface.  When taking samples from the oblong shape the small end of the shape has a much greater charge.  This is because there is a smaller surface area which leads to a much larger charge density.  \par 

In part four of this experiment we took a look at a few animations given to us online.  When given the magnitude of forces particles exert on each other they are always equal.  The force is given by Coulombs law, but you can also make the connection by thinking of newtons third law, that all forces are equal and opposite.  Sadly all of these animations have a plug-in problem so they can't be viewed leaving some questions stated in the experiment to be unanswered.  I tried to view these on multiple computers with different operating systems with no avail.  \par 

During this experiment there were a few errors that skewed our observations.  One was due to our leather wand had a flap hanging off of it causing it to not hold charge as well.  This caused one of our wands to have double the charge of the other when measuring.  This would be a systematic error in measurement because there is a defect with the equipment provided.  This made our observation skewed because one of the charges was so much higher than the other.  This was definitely the largest error in our experiment.  This experiment doesn't have any physical results, but I would classify this as a failure because the experiment didn't produce the desired results.  This could easily be fixed if we were given a proper wand.
  
%---------------------------------------------------------------------------
\singlespace
\newpage

\section*{Conclusion:}
%---------------------------------------------------------------------------
%	Conclusion
%---------------------------------------------------------------------------
In conclusion this experiment allowed us to see effects of charged particles on each other.  We did this by Measuring charged wands in different settings, paying with electrostatic spheres, and using online activities.  We don't have any theoretical or experimental values in this experiment, but I would classify this experiment as a failure.  Due to defects in equipment we did not obtain the desired results.
%--------------------------------------------------------------------------
\end{document}