%Lab 6 Report
%LabName

\documentclass{article}
\usepackage{setspace}
\usepackage[margin = 1in]{geometry}
\usepackage{graphicx}
\graphicspath{ {/home/ahaggard/Pictures/} }

%---------------------------------------------------------------------------
%	TITLE SECTION
%---------------------------------------------------------------------------
\title{Experiment 6 \\ The Oscillioscope}
\author{by \\ Austin Haggard}

\date{
	Experiment Performed: March 4 2015 \\
	Report Submitted: March 18 2015 \\[11pt]
	Lab Partners: \\ Mohit Chandiramani \\ Steve Kalingarni \\[11pt] 
	GSA: \\ C. Monroe
}
%---------------------------------------------------------------------------
\begin{document}
\maketitle
\thispagestyle{empty}
\newpage

\section*{Introduction:}
\setcounter{page}{1}
%---------------------------------------------------------------------------
%	INTRODUCTION... What we are measuring and what we are comparing it too 	/ dont use first person / audience is another student
%---------------------------------------------------------------------------
This lab was done to become familiar with the oscilloscope.  To do this we first learned about the buttons then took voltage measurements, observed waves, and even compared two waves at the same time.  We also completed an in lab worksheet which further help familiarize us with the oscilloscope. 
%---------------------------------------------------------------------------
\newpage

%---------------------------------------------------------------------------
%	DATA
%---------------------------------------------------------------------------
%	On a seperate page normally
%---------------------------------------------------------------------------

\section*{Data Analysis:}
\setcounter{page}{3}
%---------------------------------------------------------------------------
%	DATA ANALYSIS.. All equations used
%---------------------------------------------------------------------------
This experiment had no calculations.  The only information included taking measurements of voltage, wave period, and frequency.  We also have a few images of the graphs which are shown below.

\begin{figure}[h]
\caption{Square Wave}
\centering
\includegraphics[scale = .035, angle = -90]{IMG_2901} 
\end{figure}

\begin{figure}[h]
\caption{Super Postioned Waves}
\centering
\includegraphics[scale = .035, angle = -90]{IMG_2902}
\end{figure}

\begin{figure}[h]
\caption{Beats}
\centering
\includegraphics[scale= .035, angle = -90]{IMG_2903}
\end{figure}
%---------------------------------------------------------------------------
\newpage

\section*{Table of Results:}
\begin{center}
\begin{tabular}{|l|l|}
\hline
%--------------------------------------------------------------------------
%	TABLE OF RESULTS
%--------------------------------------------------------------------------
	Experimental: N/A & Theoretical: N/A  \\ \hline
	Discrepancy: N/A   & Percent Difference: N/A \\
%---------------------------------------------------------------------------
\hline
\end{tabular}
\end{center}

\section*{Discussion (Questions):}
\doublespace
%---------------------------------------------------------------------------
%	DISCUSSION
%---------------------------------------------------------------------------
%---------------------------------------------------------------------------
%	short paragraph on physics of the experiment
%---------------------------------------------------------------------------
For a DC voltage trace it should look like a straight line.  If you were to pull out the horizontal position knob the trace would look dimmer and it would stretch.  A trace would be more accurate if it has less frequency and a larger voltage.  If we were using a DC power supply instead of a signal generator it would read double the value of the signal generator.  The peak to peak voltage you measured would be the same if you were using a DC power supply or signal generator.  The phase relationship between all voltages measured in the previous procedures is that they are all equal even though some are square and some are sine waves.  

%---------------------------------------------------------------------------
%	SOURCES OF EXPERIMENTAL ERROR TO OCCUR ie.. each category, 			 		random\intrinsic and how they effected the results, and which error is 	the largest
%---------------------------------------------------------------------------
  
  
%---------------------------------------------------------------------------
%	State wether it was a success / compare theoreticle and experiemntal.  	Explain discrepencies larger than normal
%---------------------------------------------------------------------------


%---------------------------------------------------------------------------
\singlespace
\newpage

\section*{Conclusion:}
%---------------------------------------------------------------------------
%	CONCLUSION... two sentences summerize experiment.  State whether it 			was a success and compare experimental / theoretical results.
%---------------------------------------------------------------------------
This section is omitted.
%--------------------------------------------------------------------------
\end{document}